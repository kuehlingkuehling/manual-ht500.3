\section{General}

Welcome to the HT500.3 3D Printer operating manual of the Kühling\&Kühling GmbH (in the following referred to as \emph{Kühling\&Kühling}).
The HT500.3 3D Printer (in the following referred to as \emph{HT500.3} or \emph{3D Printer}) is a fully automatic stand-alone device for 
\emph{F}used-\emph{F}ilament-\emph{F}abrication (FFF) in a lab or commercial environment.
Any information needed for installing and commissioning, operating, troubleshooting, maintenance and repair of the 3D Printer are described in this document.
This user's manual must be read thoroughly as it is meant to provide the operator with all information needed to operate the HT500.3 safely and reasonably. Please always provide access to the document for any user of the 3D Printer in case of questions or problems. 

\begin{info}
  The HT500.3 3D Printer is an Open Source Hardware product. Any alteration, structural and design changes or customization for test reasons, optimization and improvement are encouraged by Kühling\&Kühling. We would like to support you with your advancements and are looking forward to receiving your feedback. We will always consider letting your achievements slip in to constantly improve the HT500.3. However, please note that Kühling\&Kühling cannot be held liable for damages and injuries resulting from such alterations 
  (see Intended use also).
\end{info}



\subsection{Valid Version}


\subsubsection{Hardware revisions}

\begin{table}[H]
  \centering
  \begin{tabulary}{\textwidth}{ L L L }
    \toprule
    Number of revision\\
    \midrule
    \hardwarerevision \\
    \bottomrule
  \end{tabulary}
\end{table}

 On the type plate at the rear side of the HT500.3 you find all information to precisely identify your 3D Printer:

\begin{itemize}
  \item Serial number
  \item Date of manufacturing
  \item Hardware Revision\footnote{You will also find the valid hardware revision in the \lbrack Setup\rbrack  menu on the touchscreen.}
\end{itemize}

\begin{figure}[H]
  \centering
  \includegraphics[width=.7\linewidth]{./img/type_plate_ht500-3.png}
  \caption{Type plate on the rear side.}
\end{figure}


\subsubsection{Software versions}

Operating sofware:
RepRapOnRails Version v2.x.x $\rightarrow$ Operating manual



\subsection{Intended use}

The HT500.3 3D Printer has been designed and built for printing three-dimensional workpieces of nearly random geometries from common 1.75 mm 
thermoplastic filament strands.
The 3D Printer features advanced hot-ends with an extrusion temperature up to 500\degree C. 
In combination with the heated build chamber (up to 70\degree C) and the heated print bed (up to 130\degree C), this makes the HT500.3 
suitable for printing a vast scope of materials. It is therefor capable of performing regular 3D printing tasks with standard thermoplastics
as well as special jobs with a variety of technical plastics.
Contact \emph{Kühling\&Kühling} or refer to our web shop for more detailed information on available and applicable materials.

The HT500.3 is intended for industrial and commercial use. It is not valid for the operation in an explosive atmosphere.
Observing this manual and adhering to the stated information is part of the proper operation.
Improper operation of the HT500.3 can lead to hazardous situations.
It is forbidden to operate the 3D Printer under conditions and for purposes other than stated in this manual.

Operating the HT500.3 is forbidden under the following circumstances: 

\begin{itemize}
  \item The HT500.3 is used for a purpose not designated.
  \item The HT500.3 or single parts are damaged, the electrical equipment has been installed improperly, or the isolation is defective.
  \item The HT500.3 does not function flawless.
  \item Mechanical components or the control system have been inexpertly altered or reconstructed.
  \item Operating parameter have been altered inadmissibly.
  \item Operation with unspecified materials.
  \item Use of unspecified tools.
  \item Failure to regularly perform the prescribed maintenance work.
  \item Operation in an explosive atmosphere.
\end{itemize}



\subsection{Warranty terms}

The general terms and conditions of Kühling\&Kühling GmbH apply. The customer is familiar with these terms latest on the day of signing the purchase contract.

The warranty terms and the liability period can be found in the contract documents and in the order confirmation.

Warranty claims and liability are voided in one or more of the following applies:

\begin{itemize}
  \item unintended usage of the HT500.3
  \item false setup, commissioning, operation repair or maintenance
  \item operating the apparatus with defective, missing, improperly installed and/or malfunctioning equipment.
  \item unauthorized or inadmissible alteration of the electrical or mechanical equipment or the operating parameters of the apparatus.
  \item use of other than the specified replacement parts, tools, and/or operating materials
  \item exceeding the specified maintenance intervals
  \item cases of disaster and force majeure
\end{itemize}

\begin{info}
  Any unintended use or structural alteration of the HT500.3 not agreed upon with and approved in writing by Kühling\&Kühling render the warranty and the EU Declaration of Conformity void and free Kühling\&Kühling from product liability. Even if approved, alterations have to be carried out by the customer thoroughly and properly. If necessary, adequate safety devices have to be installed. 
\end{info}



\subsection{Ordering wear and spare parts and material}

Wear parts must meet the technical specifications defined by \emph{Kühling\&Kühling}. \emph{Kühling\&Kühling} original parts are subject to rigid requirements and meet these standards. A complete list of available wear and spare parts and suppliable materials can be requested from \emph{Kühling\&Kühling}. 



\subsection{Imprint}

Kühling\&Kühling GmbH\\
Christianspries 30\\
24159 Kiel\\
Deutschland\\
E-Mail: office@kuehlingkuehling.de\\
Tel.: +49 (0) 431 98 35 24 73\\
\\
Local Court: Amtsgericht Kiel\\
Commercial Register No.: HRB 17535\\
Managing Directors: Jonas Kühling, Simon Kühling, Karsten Wenige\\
VAT Reg.No.: DE305873054\\
